\documentclass[10pt,twoside]{uz_kgu}
\usepackage{inputenc}
\usepackage[russian]{babel}
\newcommand{\eps}{\varepsilon}
\begin{document}
	
	\section{Результаты численных экспериментов}
	Зададим  функцию $u(x)$ с большими градиентами:
	$$u(x)=e^{-x/\varepsilon} +  \cos \frac{\pi x}{2},\ \    x\in [0,1].$$
	Вычислим $u(x+\delta)$, предполагая,  что $\delta > 0$, на основе применения классического разложения в ряд Тейлора и модифицированного разложения в ряд Тейлора.
	Классическое разложение в ряд Тейлора:
	$$u(x) \approx G_{k}(u,x) = \sum_{j=0}^{k}\frac{u^{(j)}(x_0)}{j!}(x-x_0)^j.$$
	Модифицированное разложение:
		$$u(x) \approx G_{k}(u,x) = \sum_{j=0}^{k}\frac{u^{(j)}(x_0)}{j!}(x-x_0)^j + \Big[\Phi(x) - \sum_{j=0}^{k}\frac{\Phi^{(j)}(x_0)}{j!}(x-x_0)^j \Big] \frac{u^{(k+1)}(x_0)}{\Phi^{(k+1)}(x_0)}.$$
		Оценим погрешности вычисления $u(x+\delta)$ для классической формулы Тейлора:
	
	$$\Delta_1^k =  u(x + \delta) - u(x)-\delta u'(x) - ... - \frac{\delta^k}{k!}u^{(k)}(x),$$
	и для модифицированной формулы:
	$$ \Delta_2^k = u(x + \delta) - u(x)-\delta u'(x) - ... - \frac{\delta^k}{k!}u^{(k)}(x) - \Big[\Phi(x+\delta) - \Phi(x) - \Phi^{'}(x)\delta - ... - \frac{\Phi^{(k)}(x)}{k!}\delta^k\Big] \frac{u^{(k+1)}(x)}{\Phi^{(k+1)}(x)}. $$
	Тогда для $k=1$:
	$$\Delta_1^1 = \Big|u(x + \delta) - u(x)-\delta u'(x)\Big|. $$
	
	$$ \Delta_2^1 = \Big|u(x + \delta) - u(x)-\delta u'(x) - (\Phi(x+\delta) - \Phi(x) -   \Phi^{'}(x)\delta)\frac{u^{''}(x)}{\Phi^{''}(x)} \Big|.$$
	Для $k=2$:
	
	$$\Delta_1^2 = \Big|u(x + \delta) - u(x)-\delta u'(x) - \frac{\delta^2}{2} u''(x)\Big|. $$
	
	$$ \Delta_2^2 = \Big|u(x + \delta) - u(x)-\delta u'(x)  -\frac{\delta^2}{2}u''(x) - (\Phi(x+\delta) - \Phi(x) -  \Phi^{'}(x)\delta - \Phi^{''}(x)\frac{\delta^2}{2}  )\frac{u^{'''}(x)}{\Phi^{'''}(x)} \Big|.$$
	\begin{table} [!htb]
		\caption {Погрешность вычисления $u(x+ \delta)$ в точке $x=0$ с использованием формулы Тейлора второго порядка $\Delta_1^1$ (вверху) и с использованием модифицированной формулы Тейлора $\Delta_2^1$(внизу)}
        \begin{center}
	\begin{tabular}{|c|c|c|c|c|c|c}
		\cline{1-6} $\varepsilon$ & \multicolumn{5}{c|}{$\delta$} \\
		\cline{2-6} &$10^{-1}$ & $10^{-2}$ & $10^{-3}$  & $10^{-4}$& $10^{-5}$\\
		\cline{1-6}
		$1$
		&$7.47e-03$&$7.35e-05$&$7.34e-07$&$7.34e-09$& $7.34e-11$\\
		&$3.76e-04$&$4.08e-07$&$4.11e-10$&$4.11e-13$& $4.44e-16$\\
		\cline{1-6}
		$10^{-1}$
		&$3.56e-01$&$4.71e-03$&$4.86e-05$&$4.87e-07$&$4.88e-09$\\
		&$3.23e-03$&$4.01e-06$&$4.10e-09$&$4.11e-12$&$4.22e-15$\\
		\cline{1-6}
		$10^{-2}$
		&$8.99e+00$&$3.68e-01$&$4.84e-03$&$4.98e-05$&$5.00e-07$\\
		&$1.01e-02$&$3.26e-05$&$4.01e-08$&$4.10e-11$&$4.11e-14$\\
		\cline{1-6}
		$10^{-3}$
		&$9.90e+01$&$9.00e+00$&$3.68e-01$&$4.84e-03$&$4.98e-05$\\
		&$1.21e-02$&$1.01e-04$&$3.26e-07$&$4.01e-10$&$4.10e-13$\\
		\cline{1-6}
		$10^{-4}$
		&$9.99e+02$&$9.90e+01$&$9.00e+00$&$3.68e-01$&$4.84e-03$\\
		&$1.23e-02$&$1.21e-04$&$1.01e-06$&$3.26e-09$&$4.01e-12$\\
		\cline{1-6}
	\end{tabular}
\end{center}
\end{table}
	
	
	
	

\begin{table} [!htb]
 \caption {{Погрешность вычисления $u(x+ \delta)$ в точке $x=0$ с использованием формулы Тейлора третьего порядка $\Delta_1^2$ (вверху) и с использованием модифицированной формулы Тейлора $\Delta_1^2$(внизу)}}
	\begin{center}
		\begin{tabular}{|c|c|c|c|c|c|c}
			\cline{1-6} $\varepsilon$ & \multicolumn{5}{c|}{$\delta$} \\
			\cline{2-6} &$10^{-1}$ & $10^{-2}$ & $10^{-3}$  & $10^{-4}$& $10^{-5}$\\
			\cline{1-6}
			$1$
			&$1.37e-04$&$1.64e-07$&$1.66e-10$&$1.67e-13$& $2.22e-16$\\
			&$2.53e-05$&$2.54e-09$&$2.54e-13$&$2.22e-16$& $2.22e-16$\\
			\cline{1-6}
			$10^{-1}$
			&$1.32e-01$&$1.63e-04$&$1.66e-07$&$1.67e-10$&$1.67e-13$\\
			&$2.53e-05$&$2.54e-09$&$2.54e-13$&$2.22e-16$&$2.22e-16$\\
			\cline{1-6}
			$10^{-2}$
			&$4.10e+01$&$1.32e-01$&$1.63e-04$&$1.66e-07$&$1.67e-10$\\
			&$2.53e-05$&$2.54e-09$&$2.54e-13$&$2.22e-16$&$0.00e+00$\\
			\cline{1-6}
			$10^{-3}$
			&$4.90e+03$&$4.10e+01$&$1.32e-01$&$1.63e-04$&$1.66e-07$\\
			&$2.53e-05$&$2.54e-09$&$2.54e-13$&$2.22e-16$&$2.22e-16$\\
			\cline{1-6}
			$10^{-4}$
			&$4.99e+05$&$4.90e+03$&$4.10e+01$&$1.32e-01$&$1.63e-04$\\
			&$2.53e-05$&$2.54e-09$&$2.50e-13$&$2.22e-16$&$2.22e-16$\\
			\cline{1-6}
		\end{tabular}
	\end{center}
\end{table}

При сравнении полученных результатов мы видим существенный выигрыш модифицированной формулы Тейлора при возрастании градиента решения:\\
1. Модифицированная формула Тейлора значительно точнее классической.\\
Во всех случаях (для разных значений $\varepsilon$ и $\delta$) погрешность $\Delta_\tau$ (модифицированная формула) меньше,  чем $\Delta_t$ (классическая формула).\\
2. Классическая формула Тейлора неустойчива при малых $\varepsilon.$\\
3. Модифицированная формула сохраняет точность даже для малых $\delta$.\\


\end{document}